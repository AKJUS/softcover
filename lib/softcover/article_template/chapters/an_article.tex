The is the first paragraph of the Softcover article template. It shows how to write a document in \PolyTeX, a subset of the \LaTeX\ typesetting language optimized for writing ebooks.\footnote{Pronunciations of ``LaTeX'' differ, but \emph{lay}-tech is the one I prefer.}

This is the second paragraph, showing how to \emph{emphasize} text. You can also make text \textbf{bold}.

\section{A section}
\label{sec:a_section}

This is a section. We'll take a look at some of the features supported by Softcover.

\subsection{Source code}

In plain Markdown, you can typeset code samples and other verbatim text using four spaces of indentation:

\begin{verbatim}
def hello
  puts "hello, world"
end
\end{verbatim}

Softcover also supports GitHub-style ``code fencing'' with language-specific syntax highlighting:

%= lang:ruby
\begin{code}
# "Hello, world!" in Ruby.
def hello
  puts "hello, world!"
end
\end{code}

The second of these can be combined with Softcover's \kode{codelisting} environment to make code listings via embedded \LaTeX, as shown in Listing~\ref{code:hello}.

\begin{codelisting}
\codecaption{Hello, world.}
\label{code:hello}
%= lang:ruby
\begin{code}
# "Hello, world!" in Ruby.
def hello
  puts "hello, world!"
end
\end{code}
\end{codelisting}

\subsection{Mathematics}

Softcover supports mathematical typesetting via embedded \LaTeX. This includes both inline math, such as \( \phi^2 - \phi - 1 = 0, \) and centered math, such as
\[ \phi^2 - \phi - 1 = 0. \]
Softcover also supports numbered equations via embedded \LaTeX, as seen in Eq.~\eqref{eq:phi} and Eq.~\eqref{eq:gauss}.

\begin{equation}
\label{eq:phi}
\phi = \frac{1+\sqrt{5}}{2} \approx 1.618
\end{equation}

\begin{equation}
\label{eq:gauss}
\mathbf{\nabla}\cdot\mathbf{B} = 0 \qquad\mbox{Gauss's law}
\end{equation}

\section{Images and tables}
\label{sec:images_and_tables}

This is the second section. Lorem ipsum dolor sit amet, consectetur adipisicing elit, sed do eiusmod
tempor incididunt ut labore et dolore magna aliqua. Ut enim ad minim veniam,
quis nostrud exercitation ullamco laboris nisi ut aliquip ex ea commodo
consequat. Duis aute irure dolor in reprehenderit in voluptate velit esse
cillum dolore eu fugiat nulla pariatur. Excepteur sint occaecat cupidatat non
proident, sunt in culpa qui officia deserunt mollit anim id est laborum.

\subsection{Images}

Softcover supports the inclusion of images, like this:

\image{images/figures/01_michael_hartl_headshot.jpg}

Using \LaTeX\ labels, you can also include a caption (as in Figure~\ref{fig:captioned_image}) or just a figure number (as in Figure~\ref{fig:figure_number}).

\begin{figure}[h]
\begin{center}
\image{images/figures/01_michael_hartl_headshot.jpg}
\end{center}
\caption{Some dude.\label{fig:captioned_image}}
\end{figure}

\begin{figure}[h]
\begin{center}
\image{images/figures/01_michael_hartl_headshot.jpg}
\end{center}
\caption{\label{fig:figure_number}}
\end{figure}

\subsection{Tables}

Softcover supports raw tables via the \LaTeX\ \kode{table} or \kode{longtable} environments.

\begin{longtable}{|l|l|l|l|}
\hline
\textbf{HTTP request} & \textbf{URL} & \textbf{Action} & \textbf{Purpose}\\
\kode{GET} & /users & \kode{index} & page to list all users\\
\kode{GET} & /users/1 & \kode{show} & page to show user with id \kode{1}\\
\kode{GET} & /users/new & \kode{new} & page to make a new user\\
\kode{POST} & /users & \kode{create} & create a new user\\
\kode{GET} & /users/1/edit & \kode{edit} & page to edit user with id \kode{1}\\
\kode{PATCH} & /users/1 & \kode{update} & update user with id \kode{1}\\
\kode{DELETE} & /users/1 & \kode{destroy} & delete user with id \kode{1}\\
\hline
\end{longtable}

Softcover also supports \kode{tabular} environments, as shown in Table~\ref{table:figure_placement}.

\begin{table}
\caption{Options for a float placement specifier.\label{table:figure_placement}}
\begin{tabular}{l|l}
\textbf{Specifier} & \textbf{Placement} \\ \hline
\kode{h} & Place the float \emph{approximately} here \\
\kode{h!} & Place the float \emph{(almost) exactly} here \\
\kode{t} & Place at the top of the page \\
\kode{b} & Place at the bottom of the page \\
\kode{p} & Put on a special page for floats only
\end{tabular}
\end{table}

\section{Final section}

This is the final section. The previous sections were Section~\ref{sec:a_section} and Section~\ref{sec:images_and_tables}.